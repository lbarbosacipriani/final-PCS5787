\documentclass[conference]{IEEEtran}
\IEEEoverridecommandlockouts
% The preceding line is only needed to identify funding in the first footnote. If that is unneeded, please comment it out.
\usepackage{cite}
\usepackage{amsmath,amssymb,amsfonts}
\usepackage{algorithmic}
\usepackage{graphicx}
\usepackage{textcomp}
\usepackage{xcolor}
\def\BibTeX{{\rm B\kern-.05em{\sc i\kern-.025em b}\kern-.08em
    T\kern-.1667em\lower.7ex\hbox{E}\kern-.125emX}}
\begin{document}

\title{Análise de desempenho de modelos de Aprendizado de máquina na base de dados Heart Disease*\\
{\footnotesize \textsuperscript{*}Note: Sub-titles are not captured in Xplore and
should not be used}
\thanks{Identify applicable funding agency here. If none, delete this.}
}

\author{\IEEEauthorblockN{1\textsuperscript{st} Given Name Surname}
\IEEEauthorblockA{\textit{dept. name of organization (of Aff.)} \\
\textit{name of organization (of Aff.)}\\
City, Country \\
email address or ORCID}
\and
\IEEEauthorblockN{2\textsuperscript{nd} Given Name Surname}
\IEEEauthorblockA{\textit{dept. name of organization (of Aff.)} \\
\textit{name of organization (of Aff.)}\\
City, Country \\
email address or ORCID}
\and
\IEEEauthorblockN{3\textsuperscript{rd} Given Name Surname}
\IEEEauthorblockA{\textit{dept. name of organization (of Aff.)} \\
\textit{name of organization (of Aff.)}\\
City, Country \\
email address or ORCID}
\and
\IEEEauthorblockN{4\textsuperscript{th} Given Name Surname}
\IEEEauthorblockA{\textit{dept. name of organization (of Aff.)} \\
\textit{name of organization (of Aff.)}\\
City, Country \\
email address or ORCID}
\and
\IEEEauthorblockN{5\textsuperscript{th} Given Name Surname}
\IEEEauthorblockA{\textit{dept. name of organization (of Aff.)} \\
\textit{name of organization (of Aff.)}\\
City, Country \\
email address or ORCID}
\and
\IEEEauthorblockN{6\textsuperscript{th} Given Name Surname}
\IEEEauthorblockA{\textit{dept. name of organization (of Aff.)} \\
\textit{name of organization (of Aff.)}\\
City, Country \\
email address or ORCID}
}

\maketitle

\begin{abstract}


\end{abstract}

\begin{IEEEkeywords}
Banco de dados, Aprendizado Estatístico.
\end{IEEEkeywords}

\section{Introdução}
As doenças cardíacas representam 

O objetivo do artigo é analisar a relação entre os fatores de risco associados à presença de doenças cardíacas e analisar classificadores de machine learning para realizar a identificação de doenças cardíacas. 

\section{Base de dados}

Foram utilizadas bases de dados contendo dados da cidade de Cleveland. Um total de 303 registros compõem a base. A base contém 14 atributos, sendo eles 

      -- 1. #3  (age)       
      -- 2. #4  (sex)       
      -- 3. #9  (cp)        
      -- 4. #10 (trestbps)  
      -- 5. #12 (chol)      
      -- 6. #16 (fbs)       
      -- 7. #19 (restecg)   
      -- 8. #32 (thalach)   
      -- 9. #38 (exang)     
      -- 10. #40 (oldpeak)   
      -- 11. #41 (slope)     
      -- 12. #44 (ca)        
      -- 13. #51 (thal)      
      -- 14. #58 (num)       


\section{Questões Analíticas}
Defina questões analíticas e hipoóteses sobre os
Datasets: técnicas estatísticas aplicadas à seleção e
definição de dados a serem aplicados em experimentos
computacionais.

Será analisada a relação entre a prática de atividade física e a predominância de doenças cardíacas. 
Questões levantadas.
\begin{itemize}
    \item 1.  A idade e o sexo estão relacionados com a presença de doença cardíaca?
\item 2.  Como a frequência cardíaca máxima (thalach) e o nível de colesterol (chol) variam entre pacientes com e sem doença cardíaca?
\item 3.  Existe diferença no risco de doença cardíaca entre os diferentes tipos de dor no peito (cp)?

\end{itemize}



\section{Métodos e Materiais}

Traduza as questões: em ações e procedimentos a serem
adotados em cada uma das etapas do ciclo de vida dos
dados: Planejamento, ..., Análise/Visualização/Publicação

\section{Resultados Obtidos}
Comparação entre os modelos

\section{Conclusão}
Publicação: Os trabalhos devem ser disponibilizados na
comunidade - Big Data Analytics Research Group of Escola
Politécnica da Universidade de São Paulo - Zenodo (zenodo.org).

This document is a model and instructions for \LaTeX.
Please observe the conference page limits. 


\section*{References}

Please number citations consecutively within brackets \cite{b1}. The 
sentence punctuation follows the bracket \cite{b2}. Refer simply to the reference 
number, as in \cite{b3}---do not use ``Ref. \cite{b3}'' or ``reference \cite{b3}'' except at 
the beginning of a sentence: ``Reference \cite{b3} was the first $\ldots$''

Number footnotes separately in superscripts. Place the actual footnote at 
the bottom of the column in which it was cited. Do not put footnotes in the 
abstract or reference list. Use letters for table footnotes.

Unless there are six authors or more give all authors' names; do not use 
``et al.''. Papers that have not been published, even if they have been 
submitted for publication, should be cited as ``unpublished'' \cite{b4}. Papers 
that have been accepted for publication should be cited as ``in press'' \cite{b5}. 
Capitalize only the first word in a paper title, except for proper nouns and 
element symbols.

For papers published in translation journals, please give the English 
citation first, followed by the original foreign-language citation \cite{b6}.

\begin{thebibliography}{00}
\bibitem{b1} G. Eason, B. Noble, and I. N. Sneddon, ``On certain integrals of Lipschitz-Hankel type involving products of Bessel functions,'' Phil. Trans. Roy. Soc. London, vol. A247, pp. 529--551, April 1955.
\bibitem{b2} J. Clerk Maxwell, A Treatise on Electricity and Magnetism, 3rd ed., vol. 2. Oxford: Clarendon, 1892, pp.68--73.
\bibitem{b3} I. S. Jacobs and C. P. Bean, ``Fine particles, thin films and exchange anisotropy,'' in Magnetism, vol. III, G. T. Rado and H. Suhl, Eds. New York: Academic, 1963, pp. 271--350.
\bibitem{b4} K. Elissa, ``Title of paper if known,'' unpublished.
\bibitem{b5} R. Nicole, ``Title of paper with only first word capitalized,'' J. Name Stand. Abbrev., in press.
\bibitem{b6} Y. Yorozu, M. Hirano, K. Oka, and Y. Tagawa, ``Electron spectroscopy studies on magneto-optical media and plastic substrate interface,'' IEEE Transl. J. Magn. Japan, vol. 2, pp. 740--741, August 1987 [Digests 9th Annual Conf. Magnetics Japan, p. 301, 1982].
\bibitem{b7} M. Young, The Technical Writer's Handbook. Mill Valley, CA: University Science, 1989.
\end{thebibliography}
\vspace{12pt}
\color{red}
IEEE conference templates contain guidance text for composing and formatting conference papers. Please ensure that all template text is removed from your conference paper prior to submission to the conference. Failure to remove the template text from your paper may result in your paper not being published.

\end{document}
